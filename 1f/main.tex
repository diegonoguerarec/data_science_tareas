\documentclass[a4paper,12pt]{article}
\usepackage[utf8]{inputenc}
\usepackage[spanish]{babel}
\usepackage{graphicx}
\usepackage{float}
\usepackage{amsmath}
\usepackage{url}
\usepackage{geometry}
\geometry{margin=2.5cm}

\title{Análisis del Nivel Salarial según Género y Características Laborales en Paraguay}
\author{Diego López, Diego Noguera, Juan Olmedo.\\
\small Facultad Politécnica - Universidad Nacional de Asunción}
\date{Noviembre 2025}

\begin{document}

\maketitle
\begin{center}
\textbf{Trabajo Práctico Final – Ciencia de Datos}
\end{center}

\section{Introducción}
La brecha salarial entre hombres y mujeres sigue siendo una de las principales manifestaciones de desigualdad en los mercados laborales de América Latina. Según la Comisión Económica para América Latina y el Caribe (CEPAL, 2022), las mujeres perciben en promedio un 17\% menos que los hombres por hora trabajada, incluso en ocupaciones similares. Este fenómeno tiene implicancias directas en la autonomía económica y en la distribución del bienestar social. 

Diversos estudios del Banco Mundial (2023) y de la Organización Internacional del Trabajo (OIT, 2024) destacan que las diferencias de ingreso no se explican únicamente por factores observables como la educación o la experiencia laboral, sino también por la segregación ocupacional y los estereotipos de género presentes en el mercado de trabajo. En Paraguay, la Encuesta Permanente de Hogares Continua (EPHC) permite explorar esta problemática desde un enfoque cuantitativo y reproducible, utilizando técnicas de Ciencia de Datos. 

El objetivo de este trabajo es construir un modelo de clasificación supervisada que permita predecir el nivel salarial mensual de personas empleadas a partir de variables personales y laborales, y analizar qué peso tiene el sexo frente a factores como la edad, la educación o la ocupación.

\section{Metodología}
Los datos utilizados provienen de la EPHC 2024 del Instituto Nacional de Estadística (INE), que recopila información socioeconómica de la población paraguaya. Las variables empleadas fueron: \textit{sexo (P06), edad (P02), nivel educativo (ED0504), ocupación (OCUP\_PEA), horas trabajadas (HORAB), zona (AREA), tipo de empleo (RAMA\_PEA), departamento (DPTO)} y \textit{salario neto mensual (e01aimde)}.  

El proceso de análisis incluyó las siguientes etapas:
\begin{enumerate}
    \item \textbf{Limpieza y filtrado:} se eliminaron registros sin salario declarado y se convirtieron los valores no numéricos a formato adecuado.
    \item \textbf{Creación de la variable objetivo:} se construyó \texttt{nivel\_salarial} dividiendo el salario neto en terciles (\textit{bajo}, \textit{medio} y \textit{alto}).
    \item \textbf{Codificación y partición:} se aplicó \textit{One-Hot Encoding} a las variables categóricas y se dividieron los datos en conjuntos de entrenamiento (70\%) y prueba (30\%).
    \item \textbf{Modelado:} se implementaron dos clasificadores —una \textbf{regresión logística multinomial} como modelo base y un \textbf{Random Forest} para capturar relaciones no lineales.
    \item \textbf{Evaluación:} se calcularon las métricas de \textit{accuracy}, \textit{precision}, \textit{recall} y \textit{F1-score}, además de graficar la matriz de confusión y la importancia de las variables.
\end{enumerate}

\section{Resultados}
El modelo de regresión logística alcanzó una precisión del 62\%, mientras que el Random Forest mejoró levemente con un 64\% de exactitud global. En ambos casos, las clases de salario \textit{bajo}, \textit{medio} y \textit{alto} se predijeron de forma equilibrada, aunque se observó cierta confusión entre las categorías \textit{medio} y \textit{alto}, reflejando el solapamiento natural en los niveles de ingreso.  

En términos de métricas, el modelo no lineal logró un mejor equilibrio entre \textit{precision} y \textit{recall}, especialmente para el grupo de ingresos bajos. La matriz de confusión (Figura~\ref{fig:confusion}) muestra un mayor número de aciertos en la diagonal principal, indicando una mejora general respecto al modelo lineal.

\begin{figure}[H]
    \centering
    \begin{minipage}[b]{0.45\linewidth}
        \centering
        \includegraphics[width=\linewidth]{matriz_confusion.png}
        \caption{Matriz de confusión del modelo Random Forest.}
        \label{fig:confusion}
    \end{minipage}
    \hspace{0.05\linewidth}
    \begin{minipage}[b]{0.45\linewidth}
        \centering
        \includegraphics[width=\linewidth]{importancia_variables.png}
        \caption{Importancia de las variables según el modelo Random Forest.}
        \label{fig:importancia}
    \end{minipage}
\end{figure}


En cuanto a la importancia de variables (Figura~\ref{fig:importancia}), el modelo identificó como factores más relevantes las \textbf{horas trabajadas} y la \textbf{edad}, seguidas por la \textbf{ocupación} y el \textbf{departamento}. El \textbf{género} ocupó una posición intermedia-baja, lo que sugiere que las diferencias salariales se explican principalmente por variables laborales antes que por el sexo de la persona.



\section{Discusión}
Los resultados reflejan patrones coherentes con estudios regionales. De acuerdo con CEPAL (2022) y OIT (2024), la mayor parte de la brecha salarial puede atribuirse a diferencias en las condiciones laborales y en el tipo de empleo, más que a una discriminación directa en los salarios. En este análisis, el género influye, pero en menor medida que las horas trabajadas o la edad, lo que sugiere un efecto indirecto: las mujeres suelen concentrarse en empleos con menor carga horaria o en sectores de menor remuneración.

Entre las principales limitaciones se encuentra la ausencia de variables sobre antigüedad, tipo de contrato, o responsabilidades familiares, que podrían mejorar la capacidad predictiva del modelo. Asimismo, la clasificación del salario en terciles simplifica una distribución de ingresos que en la práctica es más heterogénea.

Este tipo de análisis puede contribuir a diseñar estrategias que promuevan una participación laboral más equitativa, fomentando la inserción de mujeres en sectores de alta productividad y fortaleciendo las políticas de corresponsabilidad familiar.

\section{Conclusiones}
El estudio permitió aplicar herramientas de Ciencia de Datos para abordar una problemática social de gran relevancia. Los modelos desarrollados lograron predecir el nivel salarial con una precisión cercana al 64\%, siendo el Random Forest el método más robusto. Las variables más determinantes fueron las horas trabajadas y la edad, mientras que el género mostró un impacto menor en la predicción directa del salario. 

Sin embargo, este resultado no descarta la existencia de desigualdades, sino que resalta la necesidad de analizarlas desde un enfoque multidimensional.

\section*{Referencias}
\begin{itemize}
    \item Banco Mundial (2023). \textit{Brecha salarial de género en América Latina: Avances y desafíos}. Washington D.C.
    \item CEPAL (2022). \textit{Desigualdades de género en el mercado laboral latinoamericano}. Santiago de Chile.
    \item Instituto Nacional de Estadística (2024). \textit{Encuesta Permanente de Hogares Continua (EPHC) 2024}. Asunción, Paraguay.
    \item Organización Internacional del Trabajo (2024). \textit{Panorama Laboral en América Latina y el Caribe}. Ginebra.
    \item INE Paraguay (2025). \textit{Metodología de la EPHC}. Asunción.
\end{itemize}

\end{document}
